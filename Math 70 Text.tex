\documentclass[11pt]{report}
\usepackage{geometry}                % See geometry.pdf to learn the layout options. There are lots.
\geometry{letterpaper}                   % ... or a4paper or a5paper or ... 
%\geometry{landscape}                % Activate for for rotated page geometry
%\usepackage[parfill]{parskip}    % Activate to begin paragraphs with an empty line rather than an indent
\usepackage{graphicx}
\usepackage{amssymb}
\usepackage{epstopdf}
\usepackage{amsthm}
\usepackage{amsmath}
\usepackage{pstricks-add}
\newtheorem*{quest}{Question}
\newtheorem*{defi}{Definition}
\newcommand{\ansbox}{
\begin{center}  
\begin{tabular}{|c|}        %%%%      Big Box For Answers     %%%%%%%
\hline
\hspace{4.5in} \\
\hspace{4.5in} \\
\hspace{4.5in} \\
\hspace{4.5in} \\
\hline
\end{tabular}\\
\end{center}}
\newcommand{\Bansbox}{
\begin{center}  
\begin{tabular}{|c|}        %%%%      Big Box For Answers     %%%%%%%
\hline
\hspace{4.5in} \\
\hspace{4.5in} \\
\hspace{4.5in} \\
\hspace{4.5in} \\
\hspace{4.5in} \\
\hspace{4.5in} \\
\hspace{4.5in} \\
\hspace{4.5in} \\
\hline
\end{tabular}\\
\end{center}}

\DeclareGraphicsRule{.tif}{png}{.png}{`convert #1 `dirname #1`/`basename #1 .tif`.png}


\date{}  
                                         % Activate to display a given date or no date

\begin{document}
\title{Algebra II Workbook }
\author{For use with Math 70\\SFSU}
\maketitle
\tableofcontents
\chapter*{Chapter 0} \addcontentsline{toc}{chapter}{Chapter 0}

\section*{Beginning Stuff}\addcontentsline{toc}{section}{Beginning Stuff}




For Algebra II, we offer the time to do some review of the fundamental concepts covered in the Algebra I curriculum: 

\begin{itemize}
\item Arithmetic with fractions
\item Basics of Geometry 
\item Translating from natural language to mathematical language
\item Linear Functions
\item Solving equations in one variable
\item Vertex, Factored, and Standard form of Quadratic Polynomials
\item Basics of Linear Data Analysis
\end{itemize}

However due to lack of time, we may have to assume familiarity with these and many other concepts.  So if at any time you
feel that you have forgotten a previous topic, do not hesitate to review before moving on.  Even if you feel that you have mastered
Algebra I and pre-algebra concepts, it doesn't hurt to constantly review as we often discover new ways to think about old ideas
that help speed up our thoughts, and facilitate learning more advanced concepts.   \\

Even though much of mathematics cares little for spending time actually computing things ( we have computers to do that for us 
these days) thinking about {\it quantity} and {\it number} are the two traditional introductions into mathematical thinking.  
Mathematics is essentially the art of reasoning according to well defined, unambiguous rules of logic.  We accept these 
rules as a given.   Contradiction is the violation of these rules. Very little else is accepted without rigorous proof.   

Starting from these rules, and few other simple "givens" we can reason out the whole web of mathematical knowledge. 
In order to start dealing with numbers more effectively, we should think of them as simply well defined objects
that can interact with each other in predictable ways.

When proceeding through mathematical problems, or any problem in that case, it is crucial that we never jump to the 
next step, no matter how tiny of a leap is seems, without being completely sure that that next step is even there.  Everything
that we are doing at this level can be broken down into smaller steps that would almost seem trivially true.   Only by doing
this can we get to a point where even the big steps seem obvious to us.    Take the time
to restate the obvious at every level.   Be ready to explain to those who come after you exactly how you got to where you are.   \\
 
 				%%%%%%%%%%%%%%%%%%				%%%%%%%%%%%%%%%
				
				
 \section*{ Numbers}\addcontentsline{toc}{section}{Numbers}
 
 The most natural definition of a number, are quantities of objects.  Therefore {\bf Natural Numbers} are simply the whole numbers
 that we are most comfortable dealing with.  Using just addition, we can combine natural numbers to produce other 
 natural numbers.  
 
 However if we combine numbers by subtraction:
 $$ 10-15 = -5  $$
 
we might be left with {\bf negative quantities}.  While we might not run into negative numbers just counting apples, they are a
natural way to express many familiar concepts such as debt and temperature.  

Say I had $\$ 100 $ in my account, and spent \$ 190.  Then my balance of $-\$ 90$ is telling me that I would need to add \$ 90 
to get back to \$0.  \\

 
We'll call all the whole numbers, both positive, negative and zero, {\bf Integers}.  

We know that if we have two integers, {\bf A} and {\bf B}, whatever they are, if we add or subtract them, 
the result will still be an integer.   \\

\begin{quest}
{\Large
If {\bf A} and {\bf B} are integers, is  {\bf A} $\times$ {\bf B} always an integer?}\\
\end{quest}

\begin{quest}
{\Large
If {\bf A} and {\bf B} are integers, is {\bf A} $\div$ {\bf B}  always an integer?}\\
\end{quest}


  
We can rewrite division as 
$$ A\div B = \frac{A}{B}$$\\



\newpage


%%%%%%%%%%%%%%%%%%				%%%%%%%%%%%%%%%
				
				
\subsection*{Factoring Integers}\addcontentsline{toc}{subsection}{Factoring Integers}


The {\bf fundamental theorem of arithmetic} says that given any integer, we can write it as a product of prime numbers.   
A {\bf prime number} is simply a number that is only divisible by itself and 1.   
That is if $p$ is prime, the only factorization of $p$ is 
$ p = p\times 1$.  \\

However, if we have $N$ an integer that is not prime, then the fundamental theorem of arithmetic says that 
it can be broken up into prime factors.   

Take for example 10.    10 is not prime, since it is divisible by both 2 and 5.   Therefore we can write 
$$ 10 = 2\times 5$$
and since 2 and 5 are both prime, this is the unique factorization of 10.   

Given a larger number, the idea is the same, but the process may just take longer.  

For example 100 is still divisible by both 2 and 5.  There are no other primes that divide 100.  However 4
goes into 100 as well, as does 25.   But we can further factor these numbers.   Verify that the following is
a factorization of 100:

$$ 100= 4\times 25 = 2\times 2\times 5\times 5 = 2^2\times 5^2$$


But what if we have a large number where we can't easily recognize any factors.    The fundamental theorem
of arithmetic assures us that it is divisible by some prime number.  However it could be the case that 
the number itself is prime.   For example, can you find any factors of 541?

Don't try.  It has no factors except itself because it is a prime number. But what about 540?.   Clearly this is 
divisible by 2, since its even. And you should recognize that it is divisible by 5 and 10 (why?).   
But we want a complete factorization.  

So the most sure way to completely factor a number is to divide it successively by prime numbers, until 
we are left with only prime numbers.   
First lets make a list of some prime numbers: 
 {\Large
$$ \{ 2, 3, 5, 7, 11, 13, 17, 19, 23, 29  \} $$
}
Can you add anymore? : 
\ansbox

\begin{quest}{\Large How many prime numbers are there?  }\\
\end{quest}

Now lets try and factor 540.  First lets divide out all the 2's. Notice that we are not in any way changing 
the number, we are simply rewriting it in a different form.   And I am not showing the arithmetic, you should
do this yourself in the margin. 
\begin{align*}
540 = & {\bf 2}\cdot 270 \\
540 = &{\bf 2\cdot 2\cdot }135 \\
\end{align*}
We can't take any 2's out of 135, since it is odd.  So we go on to the next prime number, 3:

\begin{align*}
540 = & 2\cdot 2\cdot{\bf 3}\cdot 45 \\
540 = & 2\cdot 2\cdot {\bf3\cdot 3} \cdot 15 \\
540 = & 2\cdot 2\cdot {\bf3\cdot 3 \cdot 3}\cdot 5 \\
\end{align*}
Great! We can't take any 3's out of 5, but that's fine since we know that 5 is prime as well.  So now we are finished
and we have our prime factorization. 

$$ 540 = 2^2\cdot 3^3 \cdot 5$$
Note that we use exponents for brevity.  See chapter 2.


				%%%%%%%%%%%%%%%%%%				%%%%%%%%%%%%%%%
				
				
\section*{Rational Numbers}\addcontentsline{toc}{section}{Rational Numbers}
We can now build a new type of number from the integers.   
We'll call these {\bf Rational Numbers}:

\begin{defi}
\quad {\bf Rational Numbers} are all the things of the form $\frac{A}{B}$ where $A$ and $B$ are integers. 
\end{defi}

So here are some examples of rational numbers:

$$ \frac{14}{3}  \quad  \frac{17}{2}  \quad  \frac{1}{6}\quad  \frac{29}{30}  \quad  \frac{7}{6}\quad  \frac{30}{29} \quad \frac{9}{1}  \quad \frac{28}{6}     \quad \frac{391}{211} \quad \frac{1}{1009} $$

However we have to now compare these new objects in a way that makes sense.  Just like if we are given
two integers, 16 and 18, we can clearly say that 18 is more than 16.   We need to also be able to make the
same sort of statement about rational numbers like $\frac{3}{10}$ and $\frac{1}{3}$.   

First we can try and compare them to the integer 1:\\


\begin{quest}

{\Large If $\frac{A}{B} < 1$ then how should we finish this statement:
$$ A\ \underline{\hspace{.5cm}}\ B  $$}

\end{quest}


\begin{quest}

{\Large If $\frac{A}{B}>1$ then how should we finish this statement:
$$ A\ \underline{\hspace{.5cm}}\ B  $$}

\end{quest}

Now sort the above list of rational numbers into the following categories: \\

\begin{center}
\begin{tabular}{|c|c|}
\hline
& \\
 \quad {\LARGE Less than 1  } \quad  &  \quad {\LARGE Greater than 1  } \quad  \\
 & \\
 \hline
  & \\
   & \\
    & \\
     & \\
 \hline

\end{tabular}\\
\end{center}

When doing arithmetic with fractions it is always helpful to make it as simple as possible by recognizing certain
obvious equalities such as 
{\Large $$ \frac{A}{B} = A\cdot \frac{1}{B} $$} 
and 
{\Large $$ \frac{AC}{BC}= \frac{A}{B}$$}






				%%%%%%%%%%%%%%%%%%				%%%%%%%%%%%%%%%
\newpage

\section*{Linear Equations}\addcontentsline{toc}{section}{Linear Equations}

Lines on the plane are familiar objects.  But how do we actually {\it define} what a line is?   
How do we define {\it straightness ? }    Can you come up with a way to describe a  geometrically straight
curve ( a line) to someone completely unfamiliar with the concept?    Can we just draw an example? 
Any drawing will necessarily be a finite line segment, and we'll have to explain what happens past the 
edges.   

Is there some defining feature that stays true along the whole line?  Even without a frame of reference, a grid
or coordinate system, it seems that a line would still be a line.   A line just seems to pick a direction and 
go in it forever, without ever changing deviating.   The direction is {\bf constant}\\

However our interest in lines in this course is usually not very geometric.    We use them to represent 
{\bf algebraic } relationships.   That is we use the line to represent a {\bf set} of {\bf points. }\\

{\bf R\'en\'e Descartes} (1596 - 1650) is often credited with popularizing the use of what is now known as
the {\bf Cartesian coordinate system} that uses two axis, commonly a $x-$axis and $y-$axis, set at 
right angles to each other, to graphically represent ordered pairs as {\bf points}.

\begin{defi}

{\large {\bf Point on the plane }: an ordered pair $( x, y)$  ( ordered means that we can't mix up the order
that we write them in, $(x,y)\neq (y,x) $. )}

\end{defi}

To get an idea of the usefulness of this convention, lets consider the following problem.  \\

\begin{quote}I've just started working at Trader Joe's.  Currently my bank account reads \$40.00.  
 I will be starting out at \$10/hr. 
 \end{quote}
 
  Then we have the following table: \\
 
% \begin{table}[htdp]
%\begin{center}
%\caption{{\Large Can you fill out the missing entries?} }
\begin{tabular}{c|c}
 & \\
\# of hours worked (H)&Bank Balance(D)\\
\hline\\
0 & 40\\
 & \\
1 & 50\\
 & \\
2 & 60 \\
 & \\
3  & \\
 & \\
4  & \\
 & \\
\vdots  &\\
    &   110\\
    \vdots & \\
  31 & \\  
\end{tabular}\\ \\
%\end{center}
%\label{default}
%\end{table}%



It is simple to see how the balance of my account changes.   And it is obvious how the number of hours increases as well.  
But what if I wanted to display two pieces of information about a {\it single} state: {\bf how much money do I have after working $2$ 
of hours.   }  

 If we agree that the ordered pair $( H,D)$ will encode the number of hours worked as $H$ and money in the bank as $D$, then 
we can simply write  $(2,60) $, and know exactly what it means.   These pairs are already there to see on the table, but if we want
to take advantage of the intuitive graphical nature of the number line, we can plot them on the plane: \\

{\large \bf Graph the pairs of $D$ and $H$ values from the above table, but first label which axis corresponds to which. }\\

\psset{xunit=1.0cm,yunit=1.0cm}
\begin{pspicture*}(-0.75,-0.5)(10,10)
\psgrid[subgriddiv=0,gridlabels=0,gridcolor=lightgray](0,0)(-0.75,-0.5)(10,10)
\psset{xunit=0.5cm,yunit=0.1cm,algebraic=true,dotstyle=o,dotsize=3pt 0,linewidth=0.8pt,arrowsize=3pt 2,arrowinset=0.25}
\psaxes[labelFontSize=\scriptstyle,xAxis=true,yAxis=true,Dx=2,Dy=10,ticksize=-2pt 0,subticks=2]{->}(0,0)(-1.5,-5)(20,100)
\end{pspicture*}\\

\begin{quest}{\Large
What happens when we connect these points?     In what direction is the trend of my account balance 
as the number of hours increases?   }\end{quest}


\newpage
\subsection*{Graphing Linear Equations}\addcontentsline{toc}{subsection}{Graphing Linear Equations}

You should have noticed that you end up with a straight line when you connected all the points in the previous
example.  

Lets revert back to algebraic mode, and write out an equation that relates $D$ and $H$: 
\ansbox

The simplest equation you can write is of the form: 

{\Large $$ D = \text{(Money made per hour )} \cdot H + \text{ Initial amount in account} $$ }
Verify that this equation matches up with your table. \\

Now this form gives us a very convenient way of describing the {\bf direction} of the line.
We are going to describe the direction of that line in terms of the vertical movement
on the graph.  In our case, that means by how much the $D$ value, the account balance, changes.  
But there are two quantities here that are changing.  My balance only changes when the number of hours
I work changes.    So in order to be clear, we will agree that we are only concerned about what happens
to $D$ when $H$ changes by {\bf one unit}.  In our case that means one hour.\\

But how much does my balance change when I work one hour?  That is simply my hourly wage: \$10!. 
To be more explicit, let's call this {\bf the rate of change of my balance $D$ } .  In our very simply world, 
we are assuming that the only thing affecting this balance is my income.  And from our perspective, 
my balance $D$ {\bf depends} on the number of hours I work $H$.  

This is because in our above equation we are describing $D$ in terms of $H$, as some arithmetical 
operation performed on $H$.   That is my bank balance doesn't just do what it wants, it only changes
{\bf with respect to} the number of hours I work.   


So this way of describing the direction of the line that we graph is very natural, and is simply a generalization
of the other sorts of {\bf rates} that you encounter all of the time.    \\

When we talk about the graph however, sometimes we call this rate the {\bf slope}, since we are dealing
with a more geometric quantity.   However we then assume that we are graphing the 

\begin{defi} {\large  Rate of change of $X$} The amount by  
\end{defi}

How can we graph equations of the form: 
{\large $$ Ax + By = K$$ } 
\\
Determine the slope of the following linear equations:
\begin{enumerate}
{\large
\item   $ 5x -3y = 9$\\
\item $14x + 2y =0$ \\
\item $14x -2y = 0$\\
\item $ -\frac{1}{3}x - 6y = 10$\\

}
\end{enumerate}

You may have noticed a pattern.  How can we generalize a way to find the slope of a linear equation written 
in this form? 

\begin{quest}

{\Large   What is the slope of $$ Ax + By = K$$  in terms of $A, B$ and $K$.  Do we need to know what $K$ is? }\\ \\
\end{quest}


\begin{quest}

{\Large   What happens to the graph of $$ Ax + By = K$$  as we change $K$? }
\end{quest}


\newpage
To answer the last question, it is best to investigate through an example.  
Let {\large
$$ 3x+4y = K$$ }
be our line. {\bf What is the slope of the line?}  Now in order to actually graph it, we need to fix 
a $K$ value.    Below we see the line for $K=0$ and $K=4$.    \\

{\large \bf Graph the line for $K= 8$.      Now graph the line for $K=-4$.   } \\


\newrgbcolor{cqcqcq}{0.75 0.75 0.75}
\psset{xunit=2.0cm,yunit=2.0cm}
\begin{pspicture*}(-4,-3)(4,4)
\psgrid[subgriddiv=0,gridlabels=0,gridcolor=lightgray](0,0)(-4,-3)(4,4)
\psset{xunit=1.0cm,yunit=1.0cm,algebraic=true,dotstyle=o,dotsize=3pt 0,linewidth=0.8pt,arrowsize=3pt 2,arrowinset=0.25}
\psaxes[labelFontSize=\scriptstyle,xAxis=true,yAxis=true,Dx=2,Dy=2,ticksize=-2pt 0,subticks=2]{->}(0,0)(-8,-6)(8,8)
\psplot{-8}{8}{(-0-3*x)/4}
\rput[tl](0.47,-0.98){K=0}
\rput[tl](1.85,-0.92){K=4}
\psplot{-8}{8}{(--4-3*x)/4}
\end{pspicture*}

{\large \bf Draw an arrow pointing in the direction the line will move as $K$ increases 
and draw an arrow pointing in the direction the line will move as $K$ decreases. }




%%%%%%%%%%%%%%%%%%%%%% %%%%%%%%%%%%%%%%%%%%%%%%%%%%%%%%



\chapter*{Linear Inequalities}  \addcontentsline{toc}{chapter}{Linear Inequalities}

\begin{quote}
Kensin and Jorge have a space in a lunch wagon where they sell Chicken Burritos
and Hummus and Feta Wraps. They are very popular and always sell everything
they bring. This semester they have only 4 hours together to prepare. The
burritos take 10 minutes each to make and the wraps take 5 minutes each.
both the burritos and wraps have the same length, but the burritos are each 2
inches high and the wraps are each a half inch high. Their total space will take
40 inches of height. They make a profit of \$2 on each burrito and \$1.50 on each
wrap. How many of each should they make to maximize their profits?

\end{quote}

There is a lot of information contained in this situation.   We have to first start simplifying
it into terms that will let us more clearly see what is going on.   

First lets agree that {\bf $B$} will be the symbol representing the number of burritos produced in any given
business plan, and {\bf $W$ }will be the number of wraps.   Do you see how these are the two quantities 
that we are concerned with?    They can vary, but only within certain {\bf constraints}. \\
  
One of the first constraints is given in the following statement:  
\begin{quotation}{\it This semester they have only 4 hours together to prepare. The
burritos take 10 minutes each to make and the wraps take 5 minutes each.}
\end{quotation}

{\bf \large How can we now rephrase this in a succinct algebraic statement about the relationship between $B$ and $W$?\\}

First we note that for every unit of $B$, we have to account for 10 minutes of prep time, and every unit of $W$ contributes
to 5 minutes of prep time.       However in total we only have 240 minutes of available prep time ( why?   ) .   \\

Therefore we have the following relationship: 

$$ \text{\bf  Total prep time } \text{ NO MORE THAN } 240 \ min $$

But we know that the total prep time will be 10 minutes for every burrito plus 5 minutes for every wrap we make. 
Therefore {\bf Total prep time} $ = 10\cdot B + 5 \cdot W $.  \\
And lets use the following notation for the sake of space: 

$$ A \text{ NO MORE THAN } B \Rightarrow  A\leq B $$

This gives us the following algebraic statement: 
{\Large \bf
$$ 10B + 5 W \leq 240 $$}

Since both sides of the inequality are in minutes, we don't need to worry about the units.  \\

\begin{quest}
{\Large
What are the other constraints given in the problem?  Can you write similar inequalities in
terms of $B$ and $W$ for those?   \\
\ansbox

Can you come up with some pairs $(B,W)$ that will satisfy these inequalities?\\ }
\end{quest}
\Bansbox

				%%%%%%%%%%%%%%%%%%				%%%%%%%%%%%%%%%
\newpage
\section*{Graphing Inequalities } \addcontentsline{toc}{section}{Graphing Linear Inequalities}

Our goal now it try an visually represent the set of points that satisfy equations of the form: 

{\large $$ Ax + By \leq K$$ }
or 
{\large $$ Ax + By \geq K$$ }

Graphing the above type of equations will no longer give us just a line.  It will give us a whole {\bf \it region} 
on the plane: a two dimensional area that contains all the points {\it satisfying} the inequality.  \\


Lets look at the statement that we got from the first constraint in our business problem: 
{\Large \bf
$$ 10B + 5 W \leq 240 $$}

We want to visually display all of the combinations of burritos and wraps that we can make while
not taking more than 240 minutes.    However there are two other constraints that we haven't written down, 
but implicitly understand: we can't make negative numbers of burritos or wraps.  Can you 
express these as linear inequalities: 
\ansbox

Now lets draw the region that satisfies these inequalities: 

\newrgbcolor{vvvvvv}{0.33 0.33 0.33}
\psset{xunit=0.22483712022593857cm,yunit=0.2614279015614099cm,algebraic=true,dotstyle=o,dotsize=3pt 0,linewidth=0.8pt,arrowsize=3pt 2,arrowinset=0.25}
\begin{pspicture*}(-7.72,-4.6)(41.2,52.78)
\psaxes[labelFontSize=\scriptstyle,xAxis=true,yAxis=true,Dx=5,Dy=5,ticksize=-2pt 0,subticks=2]{->}(0,0)(-7.72,-4.6)(41.2,52.78)
\rput[tl](0.65,51.24){B}
\rput[tl](28.06,1.64){W}
\psplot[linecolor=vvvvvv]{-7.72}{41.2}{(--240-10*x)/5}
\end{pspicture*}\\

\begin{quest} {\Large \bf What part of the graph corresponds to all the {\it feasible} 
combinations of $B$ and $W$?   \\

Graph the other constraint that you came up with in the last section.  }
\end{quest}





%%%%%%%%%%%%%%%%%%%%%% %%%%%%%%%%%%%%%%%%%%%%%%%%%%%%%%

\chapter*{Exponents}\addcontentsline{toc}{chapter}{Exponents}


				%%%%%%%%%%%%%%%%%%				%%%%%%%%%%%%%%%
				
				
\section*{What is an exponent?}\addcontentsline{toc}{section}{What is an exponent?}

Expressions like $10 x $ simply means {\it take $x$ and add it to itself 10 times} or 
$$ 10x = x + x+ x+ x+ x+x+x+x+x +x $$
and in general 

$$ nx = \underbrace{x + x + \cdots +x}_{n \text{ times } } $$

But what if we want to {\it multiply} the thing $x$ by itself 10 times.   We can't put the 10 out 
front since we've already defined what that means.  But if we put it in a different place, then we
can save ourselves a lot of paper when writing out expressions that involve repeated multiplication. 


So lets define:

{\Large  $$ A^n = \underbrace{A \times A \times \cdots \times A}_{n \text{ times } }$$}\\

So   $2^4 = 2\times 2\times 2\times 2 = 16$ and $3^3 = 3\times 3\times 3 = 27$\\

Write out and calculate the following expressions to see if you understand:

\begin{enumerate}
\item $14^2$
\item $ 6^3 $
\item $ 10 ^5$
\item $( \frac{3}{7})^2$
\end{enumerate}

While it is not at all necessary to memorize multiplication tables or exponents, it is beneficial to be able 
to recognize when a number is a square, cube or higher power of a simpler number. For example, 
if you can see that 256  = $2^8$, that may simplify an expression that you are working with.  So 
to that end, it may be helpful to just go through at least once, all the powers of some basic numbers. 

\begin{table}[htdp]

\begin{center}
\begin{tabular}{ccccc}
$2^0= $ & \hspace{3cm} & $3^0= $& \hspace{3cm} & $5^0=$   \\
 & & & \\
$2^1= $ & \hspace{3cm} & $3^1= $& \hspace{3cm} & $5^1=$   \\
 & & &\\ 
$2^2= $ & \hspace{3cm} & $3^2= $& \hspace{3cm} & $5^2=$   \\
 & & & \\
$2^3= $ & \hspace{3cm} & $3^3= $& \hspace{3cm} & $5^3=$   \\
 & & & \\
$2^4= $ & \hspace{3cm} & $3^4= $& \hspace{3cm} & $5^4=$   \\
 & & &  \\
$2^5= $ & \hspace{3cm} & $3^5= $& \hspace{3cm} & $5^5=$   \\
 & & &  \\
$2^6= $ & \hspace{3cm} & $3^6= $& \hspace{3cm} & $5^6=$   \\
 & & &  \\
$2^7= $ & \hspace{3cm} & $3^7= $& \hspace{3cm} & $5^7=$   \\
 & & &  \\
$2^8= $ & \hspace{3cm} & $3^8= $& \hspace{3cm} & $5^8=$   \\
\end{tabular}
\end{center}
\end{table}


Using simply the definition lets derive some properties of exponents.  Write out concrete
examples to demonstrate your conclusions, but also try and show them by just using the above definition: \\
\begin{itemize}
\item
{\Large $A^n\cdot A^m =$}
\newline \newline \newline

\item 
{\Large $\frac{A^n}{ A^m} =$}
\newline \newline \newline

\item {\Large $(A^n)^m =$}
\newline \newline \newline

\item {\Large $(A^n)^m =$}
\newline \newline \newline

\item {\Large $(A\cdot B)^n =$}
\newline \newline \newline

\item {\Large $A^{-1}=$}
\newline \newline \newline

\item {\Large $A^{-n}=$}
\newline \newline \newline

\item {\Large $A^{0}=$}
\newline \newline \newline
\end{itemize}


\newpage


				%%%%%%%%%%%%%%%%%%				%%%%%%%%%%%%%%%
				
				
				
\section*{Scientific Notation and Powers of Ten}\addcontentsline{toc}{section}{Scientific Numbers and Powers of Ten}

A number in scientific notation is a number of the form 

\begin{table}[htdp]

\begin{center}
\begin{tabular}{|c|}
\hline
\\
{\Large
$  A \times 10^n$
}\\
\\
\hline
\end{tabular}
\end{center}
\label{default}
\end{table}%
where $1\leq A<10$ and $n$ is an integer.   That is $A$ is greater than or equal to 1 and less than 10.   What if $A$ is greater than 10. How can we rewrite
the following numbers in proper scientific notation: 

\begin{enumerate}
{\large
\item $10\times 10^{14}$ \\
\item $58.9  \times 10^7 $\\
\item $ 16 \times 10^{-3} $\\}
\end{enumerate}

It is very helpful to associate powers of ten with familiar ways of expressing the same number. For example, $10^2$ is just another
way of saying 100, and $10^3$ is 1000.   Fill out the following table so we can start to recognize powers of 10: 
{\Large
\begin{table}[htdp]

\begin{center}
\begin{tabular}{|c|c|c|c|}
\hline 

$n$ & \quad$10^n $\quad\quad  & \quad Metric Prefix \quad  & \quad\quad  \quad\quad Decimal (Expanded form) \quad\quad\quad\quad\quad\quad\quad\quad \\
\hline
&  & &\\
$24$ & $10^{24}$  &   &   \\
&  & &\\
$21$ & $10^{21}$  &   &   \\
&  & &\\
$18$ & $10^{18}$  &   &   \\

&  & &\\
$15$ & $10^{15}$  &   &   \\
&  & &\\
$12$ & $10^{12}$  &   &   \\
&  & &\\
$9$ & $10^{9}$  &   &   \\

&  & &\\
$6$ & $10^{6}$  &   &   \\
&  & &\\
$3$ & $10^{3}$  &   &   \\
&  & &\\
$2$ & $10^{2}$  &   &   \\
&  & &\\
$1$ & $10^{1}$  &   &   \\&  & &\\
$0$ & $10^{0}$  &   &   \\
&  & &\\
$-1$ & $10^{-1}$  &   &   \\
&  & &\\
$-2$ & $10^{-2}$  &   &   \\
&  & &\\
$-3$ & $10^{-3}$  &   &   \\
&  & &\\
$-6$ & $10^{-6}$  &   &   \\
&  & &\\

$-9$ & $10^{-9}$  &   &   \\
&  & &\\
$-12$ & $10^{-12}$  &   &   \\
&  & &\\
$-15$ & $10^{-15}$  &   &   \\
&  & &\\
$-18$ & $10^{-18}$  &   &   \\
&  & &\\
$-21$ & $10^{-21}$  &   &   \\
&  & &\\

\hline

\end{tabular}
\end{center}
\label{default}
\end{table}%
}

\newpage



				%%%%%%%%%%%%%%%%%%				%%%%%%%%%%%%%%%
\newpage

\section*{Decimal and Binary numbers}\addcontentsline{toc}{section}{Decimal and Binary numbers}







				%%%%%%%%%%%%%%%%%%				%%%%%%%%%%%%%%%
				
				
\section*{Fractional Exponents}\addcontentsline{toc}{section}{Fractional Exponents}

What you derived above are not "rules" per se.  The only rule that we started out with was the definition of the exponent.  
The above equalities are consequences of that definition, which you should be able to re-create at anytime.  
Therefore these are theorems that are true by the rules of arithmetic.  \\

Once we have figured out how the exponents behave when we preform arithmetic on the numbers being 
"exponents", we can start making sense of fractional exponents.  

For example, using just the above simplify the following expressions: 

\begin{enumerate}

\item $(2^{1/2})^2$\\
\item $( 5^{1/3} )^3$\\
\item $( 7^4)^{1/2}$\\
\end{enumerate}

What do you notice happens to the fractions?  You should remember the idea of taking the {\bf Square Root} of
a number.  We usually symbolized this by :

$$ \sqrt{X}$$.

However since we know that $\sqrt{X}^2 = X$ and that $(X^{\frac{1}{2}})^2 = X$, we have two things that equal the same thing. 
Therefore by those two things equal each other.  {\it They say exactly the same thing!} \\


\begin{table}[htdp]

\begin{center}
\begin{tabular}{|c|}
\hline
\\
{\Large
$  \sqrt{X} = X^{\frac{1}{2}}$
}\\
\\
\hline
\end{tabular}
\end{center}
\label{default}
\end{table}%

Likewise, the cube root of a number $X$, which simply means the number such that if we cube it ( multiply it by itself 3 times) we get back $X$,
is usually signified by :
$$\sqrt[3]{X}$$

However by the same argument above
$$\sqrt[3]{X} = X^{\frac{1}{3}}$$


%%%%%%%%%%%%%%%%%%%%%% %%%%%%%%%%%%%%%%%%%%%%%%%%%%%%%%


\chapter*{Function Notation}\addcontentsline{toc}{chapter}{Function Notation}
{\Large
$$f(x)  = 4x + 1$$}

Is a {\it definition} of a function we are naming $f$.  This function, by definition, takes a number as {\bf input} 
multiplies it by 4, then adds 1.    
We know that if we graphed all the pairs 
$$ (X, \text{output of $f(X)$})$$
 on the Cartesian Coordinate system we would have a line.  
 
 Find the following values where $f(x)$ is defined above: 
 $$ f(3) =  \underline{\hspace{1cm} } \quad \quad
 f(-1)   \underline{\hspace{1cm} } \quad \quad
  f(10)   \underline{\hspace{1cm} } \quad
  $$
  
  Now if we define 
  $$ g(x) = \frac{1}{2} x -3$$
  then find 
  
  $$ g(4) \underline{\hspace{1cm} } \quad$$
  
  What would happen if we {\it composed} the two functions. 
  
Can you find the value of  
$$ f(g(4)) = $$ 
just using the work you did above?
 
 
 
 
				%%%%%%%%%%%%%%%%%%				%%%%%%%%%%%%%%%
 \section*{Linear Functions}\addcontentsline{toc}{section}{Linear Functions}
 
 
 
 
				%%%%%%%%%%%%%%%%%%				%%%%%%%%%%%%%%%
 
 
 
 \section*{Exponential Functions}\addcontentsline{toc}{section}{Exponential Functions}
 
 
 Let's revisit the problem of keeping track of Alice's height : 
 \begin{quote}
 [Alice] found a little bottle�, and tied round the neck of the bottle was a paper label
with the words �DRINK ME� beautifully printed in large letters\dots\\

So Alice ventured to taste it, and, finding it very nice�, she very soon finished it off.\\

�What a curious feeling!� said Alice. �I must be shutting up like a telescope!�\\

And so it was indeed: she was now only ten inches high\dots\\

Soon her eye fell on a little glass box that was lying under the table: she opened it,
and found in it a very small cake, on which the words �EAT ME� were beautifully
marked in currants\dots\\

She ate a little bit� and very soon finished off the cake.\\

�Curiouser and curiouser!� cried Alice��Now I�m opening out like the largest
telescope that ever was! Goodbye, feet!� (for when she looked down at her feet,
they seemed to be almost out of sight, they were getting so far off�.)\\

Just at this moment her head struck against the roof of the hall.\\

SOME QUESTIONS:
In the story, whenever Alice drinks the beverage, she gets smaller, and when she
eats the cake, she gets bigger. But the author does not say how much smaller or
taller, or even how tall Alice was to start with.

Assume that for every ounce of cake Alice eats her height doubles, and for every
ounce of beverage she drinks, her height is cut in half.
 \end{quote}  
 
 To make this problem more concrete, let's fix her initial height: 
 
 {\Large $$ H_0 =  \underline{\hspace{2cm} }$$  }
 
 Lets fill in the following table of height values corresponding
 to how many ounces of cake she ate: 
 

\begin{center}
\begin{tabular}{|c|c|}
\hline
Ounces of Cakes she ate & \quad \quad  Height \quad \quad  \\
\hline
 & \\
 0 & $H_0 =\quad \quad $\\
  & \\
  \hline 
    & \\
 1     & \\
        & \\
  \hline
    & \\
2    & \\
        & \\
  \hline 
    & \\
 3     & \\
        & \\
  \hline
      & \\
     & \\
        & \\
  \hline
      & \\
   & \\
        & \\
  \hline    & \\
   & \\
        & \\
\hline    
& \\
   & \\
        & \\
  \hline& \\
   & \\
        & \\
  \hline
  
 
\end{tabular}
\end{center}


We came to the conclusion that her height is a  {\it function } of the number of ounces of cake she eats.  
Pick a letter or symbol to denote the amount of cake she ate in ounces, and fill in the following 
function: 

{\Large $$ h( \hspace{1cm} ) = H_0 \cdot 2^{( \hspace{1cm} ) }$$ }
Then write the function using your chosen initial height.   \\

There are two components to an {\bf exponential function}, the initial value, or size, 
and the growth factor. 

In this case, we have an initial height, then every time we increase the number of ounces
of cake she ate, we multiply that height by 2.   Therefore 2 is the {\bf growth factor}. \\

The following function gives us the population of a bacteria culture as a function of time $t$ in 
minutes: 
{Large
$$ P(t) = 300 \cdot (1.0443)^t$$}
\begin{quest}{\Large\bf
What size is the population when $t=0$?\\
What does that mean the initial population size is, and the growth rate? }
\end{quest}


Now fill out a table and come up with an expression for finding Alice's height
for every ounce of beverage that she drank:


\begin{center}
\begin{tabular}{|c|c|}
\hline
Ounces of Beverage she Drank & \quad \quad  Height \quad \quad  \\
\hline
 & \\
 0 & $H_0= \quad \quad $\\
  & \\
  \hline 
    & \\
 1     & \\
        & \\
  \hline
    & \\
2    & \\
        & \\
  \hline 
    & \\
 3     & \\
        & \\
  \hline
      & \\
     & \\
        & \\
  \hline
      & \\
   & \\
        & \\
  \hline    & \\
   & \\
        & \\
  \hline    & \\
   & \\
        & \\
  \hline
    & \\
   & \\
        & \\
  \hline
  
 
\end{tabular}
\end{center}



 
 \section*{Logarithmic Functions}  \addcontentsline{toc}{section}{Logarithmic Functions}
 
\begin{quest} What is an exponent? \end{quest}\Bansbox


Let $A$ and $B$ be positive real numbers and $p$ any number.  For each of the following expressions, find
an equivalent expression that only involves $\log A$ and $\log B$.  \\

\begin{quest} $\log(A\cdot B) = $ \end{quest} \Bansbox

\begin{quest} $\log(A/B) = $ \end{quest} \Bansbox

\newpage
\begin{quest} $\log(A^p) = $ \end{quest} \Bansbox

\begin{quest} $\log(1) = $ \end{quest} \Bansbox

\begin{quest} Find a way to express $\log_2(x)$, the base 2 logarithm using $\log(x)$, or the base 10 logarithm. \end{quest}\Bansbox


Determine whether the following are true:\\
\begin{enumerate}
\item $ \log(10^2 \cdot 10^3 ) = \log 10^2 + \log 10^3 $\\
\item $ \log(10^5\cdot 10^{-7} ) = \log 10^5 + \log 10^{-7} $\\
\item $\log(10^2/10^3) = \log 10^2 - \log 10^3 $\\
\item $ \log(10^3/10^{-3}) = \log 10^3/\log 10^{-3}$
\end{enumerate}

 
 
 
 
 

\end{document}  