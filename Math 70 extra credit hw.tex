\documentclass[11pt]{amsart}
\usepackage{geometry}                % See geometry.pdf to learn the layout options. There are lots.
\geometry{letterpaper}                   % ... or a4paper or a5paper or ... 
%\geometry{landscape}                % Activate for for rotated page geometry
%\usepackage[parfill]{parskip}    % Activate to begin paragraphs with an empty line rather than an indent
\usepackage{graphicx}
\usepackage{amssymb}
\usepackage{epstopdf}
\usepackage{mathtools}
\usepackage{enumerate}
\usepackage{pstricks-add}
\DeclarePairedDelimiter{\floor}{\lfloor}{\rfloor}
\newcommand{\HRule}{\rule{\linewidth}{0.5mm}}
\DeclareMathOperator{\Null}{null}
\DeclareMathOperator{\Dim}{dim}
\DeclareMathOperator{\range}{range}
\DeclareMathOperator{\lop}{\mathcal{L}}
\DeclareMathOperator{\mat}{Mat}
\DeclareMathOperator{\Span}{span}
\DeclareMathOperator{\ra}{\rangle}
\DeclareMathOperator{\la}{\langle}
\DeclareMathOperator{\Real}{Re}
\DeclareMathOperator{\R}{{\bf R}}
\DeclareMathOperator{\C}{{\bf C}}
\DeclareMathOperator{\Q}{{\bf Q}}
\DeclareMathOperator{\Z}{{\bf Z}}
\DeclareMathOperator{\vect}{{\bf v}}
\DeclareMathOperator{\x}{{\bf x}}
\DeclareMathOperator{\fix}{{Fix}}
\DeclareMathOperator{\aut}{{Aut}}
\DeclareMathOperator{\conv}{{conv}} 
\newcommand{\vlist}[1]{ {\bf v}_1, {\bf v}_2, \dots , {\bf v}_#1}
\DeclareGraphicsRule{.tif}{png}{.png}{`convert #1 `dirname #1`/`basename #1 .tif`.png}
\newcommand{\Bansbox}{
\begin{center}  
\begin{tabular}{|c|}        %%%%      Big Box For Answers     %%%%%%%
\hline
\hspace{4.5in} \\
\hspace{4.5in} \\
\hspace{4.5in} \\
\hspace{4.5in} \\
\hspace{4.5in} \\
\hspace{4.5in} \\
\hspace{4.5in} \\
\hspace{4.5in} \\
\hline
\end{tabular}\\
\end{center}}
\newcommand{\ansbox}{
\begin{center}  
\begin{tabular}{|c|}        %%%%      Big Box For Answers     %%%%%%%
\hline
\hspace{4.5in} \\
\hspace{4.5in} \\
\hspace{4.5in} \\
\hspace{4.5in} \\
\hline
\end{tabular}\\
\end{center}}
\newtheorem*{quest}{Question}
\newtheorem*{defi}{Definition}
\newcommand{\graph}{
\psset{xunit=0.788593779572266cm,yunit=0.9223730814639903cm}
\begin{pspicture*}(-9.35,-8.57)(9.67,7.7)
\psgrid[subgriddiv=0,gridlabels=0,gridcolor=lightgray](0,0)(-9.35,-8.57)(9.67,7.7)
\psset{xunit=0.788593779572266cm,yunit=0.9223730814639903cm,algebraic=true,dotstyle=o,dotsize=3pt 0,linewidth=0.8pt,arrowsize=3pt 2,arrowinset=0.25}
\psaxes[labelFontSize=\scriptstyle,xAxis=true,yAxis=true,Dx=1,Dy=1,ticksize=-2pt 0,subticks=2]{->}(0,0)(-9.35,-8.57)(9.67,7.7)
\end{pspicture*}}


%\date{}                                           % Activate to display a given date or no date

\begin{document}

\begin{minipage}{0.4\textwidth}
\begin{flushleft} \large
 \textsc{}
\end{flushleft}
\end{minipage}
\begin{minipage}{0.6\textwidth}
\begin{flushright} \Large
{\bf Math 70  \\
Extra Credit Homework}
\end{flushright}
\end{minipage}\\

\HRule

You can turn in solutions to all or some of these problems for a total of 20 extra credit points in your homework category.  
Feel free to ask questions during office hours about these problems. 



\begin{enumerate}

\item Find a number $t$ such that the line containing the points $(4,t)$ and $(-1,6)$ is perpendicular to the line that contains the points $(3,5)$
and $(1, -2)$.  \\

\item Show that the points $(-8, -65),$ $ ( 1,52) $ and $(3, 77) $ do not lie on a line.  \\


\item Show that if $x$ and $y$ are positive numbers then $$\sqrt{x+y} < \sqrt{x} + \sqrt{y} $$\\



\item For the following expressions, find a number $b$ such that the equality is true:
\begin{enumerate}
\item$ \log_b 64 = 1 $

\item$ \log_b 64 = 2 $
\item$ \log_b 64 = 3 $
\item$ \log_b 64 = 6 $
\item$ \log_b 64 = \frac{3}{2} $\\
\end{enumerate}

\item Suppose you invest \$1,000  in a bank account where interest is compounded monthly, 12 times a year.  If you have \$1,040 at the end of the first
year what was the original annual interest rate? That is what was the APR?\\

\item Given an APR, or annual percentage rate, and the number of times that interest is compounded each year $n$, write an expression that will calculate
the $AYR$, or actual yield rate.   Then use your expression to calculate the $AYR$ for the following situations: 
\begin{enumerate}
\item 9.1\% APR compounded quarterly. 
\item 4.6\% APR compound monthly. 
\item 15.9\% APR compounded weekly. \\
\end{enumerate}


\item Suppose that $f(x)$ is a quadratic function whose vertex is at $(3,2)$. Let $g(x) = 4x+ 5$.  What
are the coordinates of the vertex of the graph of $g\circ f $? \\


\item   Suppose that $f(x)  = ax^2 + bx + c$ where $a\neq 0$.    Show that the vertex of a graph of $f$
is the point $( -\frac{b}{2a} , \frac{4ac - b^2 }{ 4a} )$


\end{enumerate}








\end{document}  