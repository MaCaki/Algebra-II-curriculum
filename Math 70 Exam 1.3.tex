\documentclass[11pt]{amsart}
\usepackage{geometry}                % See geometry.pdf to learn the layout options. There are lots.
\geometry{letterpaper}                   % ... or a4paper or a5paper or ... 
%\geometry{landscape}                % Activate for for rotated page geometry
%\usepackage[parfill]{parskip}    % Activate to begin paragraphs with an empty line rather than an indent
\usepackage{graphicx}
\usepackage{amssymb}
\usepackage{epstopdf}
\usepackage{enumerate}
\usepackage{pstricks-add}

\DeclareGraphicsRule{.tif}{png}{.png}{`convert #1 `dirname #1`/`basename #1 .tif`.png}

\title{Math 60: Unit Assessment}

%\date{}                                           % Activate to display a given date or no date

\begin{document}
%\maketitle

\section*{\bf  Math 70: Exam 1 }
%%%%%%%%%%%%%%%%%%%%%%%%%%%%%%%%%%%%%%%%%%%%%%%

Simplify the following expressions as much as possible: \\

\begin{enumerate}

\item $\sqrt{12x^2}$\\ \\ \\ \\ 




\item $ (\sqrt[3]{z})^{6} $ \\ \\ \\ \\ 

 \item $\sqrt{28x} $ \\ \\ \\ \\ 
 \item $\sqrt{81x^4 }$\\ \\ \\ \\ 
\item $ (32 x^5)^{1/5 }$\\ \\ \\ \\ 
 
  \item{\Large $\frac{7.5\times 10^10}{5\times 10^3} $} \\ \\ \\ \\ \\
  
    \item $ 10^{-9}\cdot 10^{2} $ \\ \\ \\ \\ \\
    
        \item $ 10^{12}\cdot 10^{3} $ \\ \\ \\ \\ \\

\end{enumerate}


\subsection*{Graph the following inequality :}
$$x-4y  \geq 8 $$ \\


\psset{xunit=0.788593779572266cm,yunit=0.9223730814639903cm}
\begin{pspicture*}(-9.35,-8.57)(9.67,7.7)
\psgrid[subgriddiv=0,gridlabels=0,gridcolor=lightgray](0,0)(-9.35,-8.57)(9.67,7.7)
\psset{xunit=0.788593779572266cm,yunit=0.9223730814639903cm,algebraic=true,dotstyle=o,dotsize=3pt 0,linewidth=0.8pt,arrowsize=3pt 2,arrowinset=0.25}
\psaxes[labelFontSize=\scriptstyle,xAxis=true,yAxis=true,Dx=1,Dy=1,ticksize=-2pt 0,subticks=2]{->}(0,0)(-9.35,-8.57)(9.67,7.7)
\end{pspicture*}

\newpage




\subsection*{Graph the following inequalities on a single graph, and find the vertices of the resulting region. }
$$ \left\{  \begin{array}{cc}
    y-2x& \leq 6 \\
   3x + y   &\leq 6 \\
    y  & \geq 0 
  \end{array} \right.$$

\psset{xunit=0.788593779572266cm,yunit=0.9223730814639903cm}
\begin{pspicture*}(-9.35,-8.57)(9.67,7.7)
\psgrid[subgriddiv=0,gridlabels=0,gridcolor=lightgray](0,0)(-9.35,-8.57)(9.67,7.7)
\psset{xunit=0.788593779572266cm,yunit=0.9223730814639903cm,algebraic=true,dotstyle=o,dotsize=3pt 0,linewidth=0.8pt,arrowsize=3pt 2,arrowinset=0.25}
\psaxes[labelFontSize=\scriptstyle,xAxis=true,yAxis=true,Dx=1,Dy=1,ticksize=-2pt 0,subticks=2]{->}(0,0)(-9.35,-8.57)(9.67,7.7)
\end{pspicture*}



\newpage

{\bf You are sent to the market to purchase flour for a bakery for the week.    Storage at the baker can handle at most 300 more lbs. of flour.  However there are 
two kinds of flour you can buy: white and whole wheat.   The baker needs at least 145 lbs of whole wheat flour this week, and at least 175 lbs. of white flour.  }\\
\begin{enumerate}[a]
\item  {\it \large Write a system of inequalities describing the above constraints using  $X$ to represent the amount of white flour you purchase, and $Y$ to represent
the amount of whole wheat flour.   }  \\ \\ \\ \\ \\ \\ \\ \\ \\


\newpage
\item {\ Graph the feasible region described by those inequalities and label the coordinates of the vertices. }\\
\psset{xunit=0.9541984732824428cm,yunit=1.1160714285714282cm,algebraic=true,dotstyle=o,dotsize=3pt 0,linewidth=0.8pt,arrowsize=3pt 2,arrowinset=0.25}
\begin{pspicture*}(-0.96,-0.96)(14.76,12.48)
\psaxes[labelFontSize=\scriptstyle,xAxis=true,yAxis=true,Dx=100,Dy=100,ticksize=-0pt 0,subticks=0]{->}(0,0)(-0.96,-0.96)(14.76,12.48)
\end{pspicture*}\\ 



\item{ Find three points that are in the feasible region } \\ \\ \\ \\ 




\item{ \it {\bf Extra Credit}  If whole wheat flour costs \$0.50/lb. and white flour costs \$0.75/lb find the cheapest combination and the most 
expensive combination within the feasible region.   } 
\end{enumerate}








\end{document}  