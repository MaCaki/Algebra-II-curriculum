\documentclass[11pt]{amsart}
\usepackage{geometry}                % See geometry.pdf to learn the layout options. There are lots.
\geometry{letterpaper}                   % ... or a4paper or a5paper or ... 
%\geometry{landscape}                % Activate for for rotated page geometry
%\usepackage[parfill]{parskip}    % Activate to begin paragraphs with an empty line rather than an indent
\usepackage{graphicx}
\usepackage{amssymb}
\usepackage{epstopdf}
\usepackage{mathtools}
\usepackage{enumerate}
\DeclarePairedDelimiter{\floor}{\lfloor}{\rfloor}
\newcommand{\HRule}{\rule{\linewidth}{0.5mm}}
\DeclareMathOperator{\Null}{null}
\DeclareMathOperator{\Dim}{dim}
\DeclareMathOperator{\range}{range}
\DeclareMathOperator{\lop}{\mathcal{L}}
\DeclareMathOperator{\mat}{Mat}
\DeclareMathOperator{\Span}{span}
\DeclareMathOperator{\ra}{\rangle}
\DeclareMathOperator{\la}{\langle}
\DeclareMathOperator{\Real}{Re}
\DeclareMathOperator{\R}{{\bf R}}
\DeclareMathOperator{\C}{{\bf C}}
\DeclareMathOperator{\Q}{{\bf Q}}
\DeclareMathOperator{\Z}{{\bf Z}}
\DeclareMathOperator{\vect}{{\bf v}}
\DeclareMathOperator{\x}{{\bf x}}
\DeclareMathOperator{\fix}{{Fix}}
\DeclareMathOperator{\aut}{{Aut}}
\DeclareMathOperator{\conv}{{conv}} 
\newcommand{\vlist}[1]{ {\bf v}_1, {\bf v}_2, \dots , {\bf v}_#1}
\DeclareGraphicsRule{.tif}{png}{.png}{`convert #1 `dirname #1`/`basename #1 .tif`.png}
\newcommand{\Bansbox}{
\begin{center}  
\begin{tabular}{|c|}        %%%%      Big Box For Answers     %%%%%%%
\hline
\hspace{4.5in} \\
\hspace{4.5in} \\
\hspace{4.5in} \\
\hspace{4.5in} \\
\hspace{4.5in} \\
\hspace{4.5in} \\
\hspace{4.5in} \\
\hspace{4.5in} \\
\hline
\end{tabular}\\
\end{center}}
\newcommand{\ansbox}{
\begin{center}  
\begin{tabular}{|c|}        %%%%      Big Box For Answers     %%%%%%%
\hline
\hspace{4.5in} \\
\hspace{4.5in} \\
\hspace{4.5in} \\
\hspace{4.5in} \\
\hline
\end{tabular}\\
\end{center}}
\newtheorem*{quest}{Question}
\newtheorem*{defi}{Definition}

%\date{}                                           % Activate to display a given date or no date

\begin{document}

\begin{minipage}{0.4\textwidth}
\begin{flushleft} \large
 \textsc{April 23, 2012}
\end{flushleft}
\end{minipage}
\begin{minipage}{0.6\textwidth}
\begin{flushright} \Large
{\bf Math 70   \\
Exam 2}
\end{flushright}
\end{minipage}\\

\HRule
\subsection*{1}
The base 10 logarithm function is defined by the following equivalent statements: 
$$ \log(x) = y \quad \Leftrightarrow \quad 10^y = x $$
Use this definition to prove the following properties of logarithms: 
\begin{enumerate}[a]
\item $\log(A\cdot B) = \log(A) + \log(B) $
\item $\log(A/B) = \log(A) - \log(B) $
\item $ \log(A^p) = p \log(A)$
\end{enumerate}




\newpage

\subsection*{2}   % simplify logarithmic expressions and solve  exponential equations
{\bf Solve for $x$:} 
\begin{enumerate}[a]
\item $10^{4x} = 100,000,000$\\ \\ \\ 
\item $5\cdot 10^{6-x} = 0.005 $ \\ \\ \\ 
\item $\log(4/2) = 2 $ \\ \\ \\
\item $ \log(10^x)= 88.1 $ \\ \\ \\
\end{enumerate}

{\bf Expand and simplify as much as possible:} 
\begin{enumerate}[a]
\item $\displaystyle \log \left(\frac{8x^2}{\sqrt[3]{y}}\right) $\\ \\ \\
\item $\log(1000\frac{x}{y}) $ \\ \\ \\
\end{enumerate}

{\bf Give a numerical answer : } 
\begin{enumerate}[a]
\item $2\log(2) + \log(5) $\\ \\ \\
\item $\log(300) - \log( 3) $ \\ \\ \\ 
\item $\log(\frac{3}{100}) - \log(3) $ \\ \\ \\ 
\end{enumerate}




\newpage
\subsection*{3} % exponential growth function What is percent increase.  
After observing the mold growing on the wall of bathroom, you conclude that it seems to double in size every month.  \\
{\bf a} If it is measured to take up 3 square feet at the beginning of this month, write an exponential function that keeps track
of the size of the mold given the number of months from now. \\
{\bf b} The wall is 9 by 11 feet.  How long until the mold completely covers the wall assuming that it doubles in size consistently. 
Give an estimate as well as an exact answer in terms of logarithms. 
\vspace{7cm}
\newpage
\subsection*{4} % exponential decay (half-life function) 
The half life of a radioactive element X is 8 months. That is after 8 months a sample of X will loose half its mass.\\
{\bf a} If we start out with 1280 grams, how much will be left after 4 years? \\
{\bf b} Write an exponential decay equation for the half life of this element where the dependent variable $t$ is measured in 8 month intervals and given the above
initial amount. \\
{\bf c} Write an exponential decay equation where $t$ is measured in {\bf one} month intervals and give the same initial amount as above. 



\newpage
\subsection*{5} Let $n$ be the number of times the interest is compounded each year,
and $r$ the annual interest rate (APR), and $P_0$ the principal investment.  Then the amount you will have after $t$ number of years is
$$ P(t) = P_0 \left( 1+\frac{r}{n}\right)^{nt} $$
\begin{enumerate}[a]
\item Write the equation that tracks the value of \$10,000 put into an investment with a 12\% annual rate compounded once every other month. 
\item If that same investment's interest was compounded twice a year, what would the corresponding equation be? Find the {\bf effective} interest
rate. That is how much interest you actually earn after one year. 
\end{enumerate}

\vspace{7cm}

\subsection*{6} % continuously compounded interest - solve for the rate
If a principal $P_0$ is invested at an annual rate of $r$ ( expressed as a decimal not as a percent) and is compounded 
continuously then its value for any given time $t$ measured in years is 
$$P(t) = P_0e^{rt}$$
\begin{enumerate}[a]
\item \$100 is invested at a continuously compounded rate. After 5 years we have \$200.  Find the original annual rate
it was invested at.  ( You can express your answer in terms of the natural log, $\ln$, or base-$e$ logarithm. ) \
\end{enumerate}




\end{document}  